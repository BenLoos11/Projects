% Options for packages loaded elsewhere
\PassOptionsToPackage{unicode}{hyperref}
\PassOptionsToPackage{hyphens}{url}
%
\documentclass[
]{article}
\usepackage{amsmath,amssymb}
\usepackage{iftex}
\ifPDFTeX
  \usepackage[T1]{fontenc}
  \usepackage[utf8]{inputenc}
  \usepackage{textcomp} % provide euro and other symbols
\else % if luatex or xetex
  \usepackage{unicode-math} % this also loads fontspec
  \defaultfontfeatures{Scale=MatchLowercase}
  \defaultfontfeatures[\rmfamily]{Ligatures=TeX,Scale=1}
\fi
\usepackage{lmodern}
\ifPDFTeX\else
  % xetex/luatex font selection
\fi
% Use upquote if available, for straight quotes in verbatim environments
\IfFileExists{upquote.sty}{\usepackage{upquote}}{}
\IfFileExists{microtype.sty}{% use microtype if available
  \usepackage[]{microtype}
  \UseMicrotypeSet[protrusion]{basicmath} % disable protrusion for tt fonts
}{}
\makeatletter
\@ifundefined{KOMAClassName}{% if non-KOMA class
  \IfFileExists{parskip.sty}{%
    \usepackage{parskip}
  }{% else
    \setlength{\parindent}{0pt}
    \setlength{\parskip}{6pt plus 2pt minus 1pt}}
}{% if KOMA class
  \KOMAoptions{parskip=half}}
\makeatother
\usepackage{xcolor}
\usepackage[margin=1in]{geometry}
\usepackage{color}
\usepackage{fancyvrb}
\newcommand{\VerbBar}{|}
\newcommand{\VERB}{\Verb[commandchars=\\\{\}]}
\DefineVerbatimEnvironment{Highlighting}{Verbatim}{commandchars=\\\{\}}
% Add ',fontsize=\small' for more characters per line
\usepackage{framed}
\definecolor{shadecolor}{RGB}{248,248,248}
\newenvironment{Shaded}{\begin{snugshade}}{\end{snugshade}}
\newcommand{\AlertTok}[1]{\textcolor[rgb]{0.94,0.16,0.16}{#1}}
\newcommand{\AnnotationTok}[1]{\textcolor[rgb]{0.56,0.35,0.01}{\textbf{\textit{#1}}}}
\newcommand{\AttributeTok}[1]{\textcolor[rgb]{0.13,0.29,0.53}{#1}}
\newcommand{\BaseNTok}[1]{\textcolor[rgb]{0.00,0.00,0.81}{#1}}
\newcommand{\BuiltInTok}[1]{#1}
\newcommand{\CharTok}[1]{\textcolor[rgb]{0.31,0.60,0.02}{#1}}
\newcommand{\CommentTok}[1]{\textcolor[rgb]{0.56,0.35,0.01}{\textit{#1}}}
\newcommand{\CommentVarTok}[1]{\textcolor[rgb]{0.56,0.35,0.01}{\textbf{\textit{#1}}}}
\newcommand{\ConstantTok}[1]{\textcolor[rgb]{0.56,0.35,0.01}{#1}}
\newcommand{\ControlFlowTok}[1]{\textcolor[rgb]{0.13,0.29,0.53}{\textbf{#1}}}
\newcommand{\DataTypeTok}[1]{\textcolor[rgb]{0.13,0.29,0.53}{#1}}
\newcommand{\DecValTok}[1]{\textcolor[rgb]{0.00,0.00,0.81}{#1}}
\newcommand{\DocumentationTok}[1]{\textcolor[rgb]{0.56,0.35,0.01}{\textbf{\textit{#1}}}}
\newcommand{\ErrorTok}[1]{\textcolor[rgb]{0.64,0.00,0.00}{\textbf{#1}}}
\newcommand{\ExtensionTok}[1]{#1}
\newcommand{\FloatTok}[1]{\textcolor[rgb]{0.00,0.00,0.81}{#1}}
\newcommand{\FunctionTok}[1]{\textcolor[rgb]{0.13,0.29,0.53}{\textbf{#1}}}
\newcommand{\ImportTok}[1]{#1}
\newcommand{\InformationTok}[1]{\textcolor[rgb]{0.56,0.35,0.01}{\textbf{\textit{#1}}}}
\newcommand{\KeywordTok}[1]{\textcolor[rgb]{0.13,0.29,0.53}{\textbf{#1}}}
\newcommand{\NormalTok}[1]{#1}
\newcommand{\OperatorTok}[1]{\textcolor[rgb]{0.81,0.36,0.00}{\textbf{#1}}}
\newcommand{\OtherTok}[1]{\textcolor[rgb]{0.56,0.35,0.01}{#1}}
\newcommand{\PreprocessorTok}[1]{\textcolor[rgb]{0.56,0.35,0.01}{\textit{#1}}}
\newcommand{\RegionMarkerTok}[1]{#1}
\newcommand{\SpecialCharTok}[1]{\textcolor[rgb]{0.81,0.36,0.00}{\textbf{#1}}}
\newcommand{\SpecialStringTok}[1]{\textcolor[rgb]{0.31,0.60,0.02}{#1}}
\newcommand{\StringTok}[1]{\textcolor[rgb]{0.31,0.60,0.02}{#1}}
\newcommand{\VariableTok}[1]{\textcolor[rgb]{0.00,0.00,0.00}{#1}}
\newcommand{\VerbatimStringTok}[1]{\textcolor[rgb]{0.31,0.60,0.02}{#1}}
\newcommand{\WarningTok}[1]{\textcolor[rgb]{0.56,0.35,0.01}{\textbf{\textit{#1}}}}
\usepackage{graphicx}
\makeatletter
\newsavebox\pandoc@box
\newcommand*\pandocbounded[1]{% scales image to fit in text height/width
  \sbox\pandoc@box{#1}%
  \Gscale@div\@tempa{\textheight}{\dimexpr\ht\pandoc@box+\dp\pandoc@box\relax}%
  \Gscale@div\@tempb{\linewidth}{\wd\pandoc@box}%
  \ifdim\@tempb\p@<\@tempa\p@\let\@tempa\@tempb\fi% select the smaller of both
  \ifdim\@tempa\p@<\p@\scalebox{\@tempa}{\usebox\pandoc@box}%
  \else\usebox{\pandoc@box}%
  \fi%
}
% Set default figure placement to htbp
\def\fps@figure{htbp}
\makeatother
\setlength{\emergencystretch}{3em} % prevent overfull lines
\providecommand{\tightlist}{%
  \setlength{\itemsep}{0pt}\setlength{\parskip}{0pt}}
\setcounter{secnumdepth}{-\maxdimen} % remove section numbering
\usepackage{bookmark}
\IfFileExists{xurl.sty}{\usepackage{xurl}}{} % add URL line breaks if available
\urlstyle{same}
\hypersetup{
  pdftitle={Ben Loos Final Project STAT 600},
  hidelinks,
  pdfcreator={LaTeX via pandoc}}

\title{Ben Loos Final Project STAT 600}
\author{}
\date{\vspace{-2.5em}}

\begin{document}
\maketitle

The problem comes from data that is received from an agricultural field
for crop performance. While there can be many factors that can
contribute to how crops perform, we want to look at the seeding rate and
if that has anything to do with the growth rate for certain crops. The
reason we want to look at this variable is because it can be a great
indicator for farmers' how if it can greatly impact on the productivity
of the harvest. The goal of this analysis is to figure out if we can
show how the data from recent years can help with the most recent data
on past yields, and the current yields. The way we are going to do this
is by aggregating the yield and seeding data in 50 m x 50m grid cells,
which should help reduce noise and help create spatial framework. After
we gather data on this, we will turn to normalizing the data to account
for the different units across crops such as soybean yield being 60
bu/acre while corn yield is 180 bu/acre. This will allow for better
comparison between the crops and will allow for better analysis on the
question of whether the seeding rate can indicate how the yield will
turn out. The variables that we will use from these data sets are the
`Northing' and `Easting' variable to help organize and sort the data
with these variables indciating the current location of the plots. Then
we will use the `yield' and `AppliedRate' variables as our parameters
for conducting our analysis and plotting.

\#\#Grid - The grid will become the main focal point. We want to create
the data in a grid to help show how the plots look somewhat like how
they would be interpreted in the field. In addition, we will create a
variable called ``Cell'' which will help distribute the data so that we
can combine sections of the same field into the plot and make geospatial
easier to interpret. We will use the `Northing' and `Easting' variables
to help create the variable.

\begin{Shaded}
\begin{Highlighting}[]
\CommentTok{\#read in the files }
\NormalTok{soyharvest17 }\OtherTok{\textless{}{-}} \FunctionTok{read.csv}\NormalTok{(}\StringTok{"C:/Users/DSU/Downloads/A 2017 Soybeans Harvest.csv"}\NormalTok{)}
\NormalTok{ cornharvest18}\OtherTok{\textless{}{-}} \FunctionTok{read.csv}\NormalTok{(}\StringTok{"C:/Users/DSU/Downloads/A 2018 Corn Harvest.csv"}\NormalTok{)}
\NormalTok{cornseed18 }\OtherTok{\textless{}{-}}\FunctionTok{read.csv}\NormalTok{(}\StringTok{"C:/Users/DSU/Downloads/A 2018 Corn Seeding.csv"}\NormalTok{)}
\NormalTok{soyharvest19 }\OtherTok{\textless{}{-}} \FunctionTok{read.csv}\NormalTok{(}\StringTok{"C:/Users/DSU/Downloads/A 2019 Soybeans Harvest.csv"}\NormalTok{)}
\NormalTok{cornharvest20 }\OtherTok{\textless{}{-}} \FunctionTok{read.csv}\NormalTok{(}\StringTok{"C:/Users/DSU/Downloads/A 2020 Corn Harvest.csv"}\NormalTok{)}
\NormalTok{cornseed20 }\OtherTok{\textless{}{-}} \FunctionTok{read.csv}\NormalTok{(}\StringTok{"C:/Users/DSU/Downloads/A 2020 Corn Seeding.csv"}\NormalTok{)}

\CommentTok{\#create function for the spatial grid }
\NormalTok{gridfuction }\OtherTok{\textless{}{-}} \ControlFlowTok{function}\NormalTok{(gridcells) \{}
  \CommentTok{\#the row variable should help define the northing variable into the grid and help with creating cell variable }
\NormalTok{  gridcells}\SpecialCharTok{$}\NormalTok{row }\OtherTok{\textless{}{-}} \FunctionTok{ceiling}\NormalTok{(gridcells}\SpecialCharTok{$}\NormalTok{Northing }\SpecialCharTok{/} \DecValTok{50}\NormalTok{)}
  \CommentTok{\#Column variable goes with the cell variable but using the Easting variable}
\NormalTok{  gridcells}\SpecialCharTok{$}\NormalTok{column }\OtherTok{\textless{}{-}} \FunctionTok{ceiling}\NormalTok{(gridcells}\SpecialCharTok{$}\NormalTok{Easting }\SpecialCharTok{/} \DecValTok{50}\NormalTok{)}
  \CommentTok{\#Creating the cell variable to help identify spatial data with plots }
\NormalTok{  gridcells}\SpecialCharTok{$}\NormalTok{cell }\OtherTok{\textless{}{-}}\NormalTok{ gridcells}\SpecialCharTok{$}\NormalTok{row }\SpecialCharTok{*} \DecValTok{1000} \SpecialCharTok{+}\NormalTok{ gridcells}\SpecialCharTok{$}\NormalTok{column}
  \FunctionTok{return}\NormalTok{(gridcells)}
\NormalTok{\}}
\CommentTok{\#convert the normal data frame to the grid and create variable row, column, and the identifying variable cell }
\NormalTok{soyharvest17  }\OtherTok{\textless{}{-}} \FunctionTok{gridfuction}\NormalTok{(soyharvest17)}
\NormalTok{cornharvest18 }\OtherTok{\textless{}{-}} \FunctionTok{gridfuction}\NormalTok{(cornharvest18)}
\NormalTok{soyharvest19  }\OtherTok{\textless{}{-}} \FunctionTok{gridfuction}\NormalTok{(soyharvest19)}
\NormalTok{cornharvest20 }\OtherTok{\textless{}{-}} \FunctionTok{gridfuction}\NormalTok{(cornharvest20)}
\NormalTok{cornseed18 }\OtherTok{\textless{}{-}} \FunctionTok{gridfuction}\NormalTok{(cornseed18)}
\NormalTok{cornseed20 }\OtherTok{\textless{}{-}} \FunctionTok{gridfuction}\NormalTok{(cornseed20)}
\end{Highlighting}
\end{Shaded}

\begin{Shaded}
\begin{Highlighting}[]
\CommentTok{\#Showcase of how the grid looks at a singular data set bases}
\FunctionTok{plot}\NormalTok{(row }\SpecialCharTok{\textasciitilde{}}\NormalTok{ column,}\AttributeTok{data=}\NormalTok{cornseed20)}
\FunctionTok{abline}\NormalTok{(}\AttributeTok{h=}\DecValTok{1}\SpecialCharTok{:}\DecValTok{12}\FloatTok{+0.5}\NormalTok{,}\AttributeTok{v=}\DecValTok{1}\SpecialCharTok{:}\DecValTok{20}\FloatTok{+0.5}\NormalTok{,}\AttributeTok{col=}\StringTok{\textquotesingle{}red\textquotesingle{}}\NormalTok{)}
\end{Highlighting}
\end{Shaded}

\pandocbounded{\includegraphics[keepaspectratio]{Ben.Loos.Final_files/figure-latex/unnamed-chunk-2-1.pdf}}
\#\# Aggregate data

\begin{Shaded}
\begin{Highlighting}[]
\CommentTok{\#aggregation of each data frame and produce a variable that fits the final data format}
\NormalTok{soyharvest17agg }\OtherTok{\textless{}{-}} \FunctionTok{aggregate}\NormalTok{(Yield }\SpecialCharTok{\textasciitilde{}}\NormalTok{ cell, }\AttributeTok{data =}\NormalTok{ soyharvest17, mean)}
\NormalTok{soyharvest17agg}\SpecialCharTok{$}\NormalTok{Y17 }\OtherTok{\textless{}{-}}\NormalTok{ soyharvest17agg}\SpecialCharTok{$}\NormalTok{Yield}
\NormalTok{cornharvest18agg }\OtherTok{\textless{}{-}} \FunctionTok{aggregate}\NormalTok{(Yield }\SpecialCharTok{\textasciitilde{}}\NormalTok{ cell, }\AttributeTok{data =}\NormalTok{ cornharvest18, mean)}
\NormalTok{cornharvest18agg}\SpecialCharTok{$}\NormalTok{Y18 }\OtherTok{\textless{}{-}}\NormalTok{ cornharvest18agg}\SpecialCharTok{$}\NormalTok{Yield}
\NormalTok{cornseed18agg }\OtherTok{\textless{}{-}} \FunctionTok{aggregate}\NormalTok{(AppliedRate }\SpecialCharTok{\textasciitilde{}}\NormalTok{ cell, }\AttributeTok{data =}\NormalTok{ cornseed18, mean)}
\NormalTok{cornseed18agg}\SpecialCharTok{$}\NormalTok{AR18 }\OtherTok{\textless{}{-}}\NormalTok{ cornseed18agg}\SpecialCharTok{$}\NormalTok{AppliedRate}
\NormalTok{soyharvest19agg }\OtherTok{\textless{}{-}} \FunctionTok{aggregate}\NormalTok{(Yield }\SpecialCharTok{\textasciitilde{}}\NormalTok{ cell, }\AttributeTok{data =}\NormalTok{ soyharvest19, mean)}
\NormalTok{soyharvest19agg}\SpecialCharTok{$}\NormalTok{Y19 }\OtherTok{\textless{}{-}}\NormalTok{ soyharvest19agg}\SpecialCharTok{$}\NormalTok{Yield}
\NormalTok{cornharvest20agg }\OtherTok{\textless{}{-}} \FunctionTok{aggregate}\NormalTok{(Yield }\SpecialCharTok{\textasciitilde{}}\NormalTok{ cell, }\AttributeTok{data =}\NormalTok{ cornharvest20, mean)}
\NormalTok{cornharvest20agg}\SpecialCharTok{$}\NormalTok{Y20 }\OtherTok{\textless{}{-}}\NormalTok{ cornharvest20agg}\SpecialCharTok{$}\NormalTok{Yield}
\NormalTok{cornseed20agg }\OtherTok{\textless{}{-}} \FunctionTok{aggregate}\NormalTok{(AppliedRate }\SpecialCharTok{\textasciitilde{}}\NormalTok{ cell, }\AttributeTok{data =}\NormalTok{ cornseed20 , mean)}
\NormalTok{cornseed20agg}\SpecialCharTok{$}\NormalTok{AR20 }\OtherTok{\textless{}{-}}\NormalTok{ cornseed20agg}\SpecialCharTok{$}\NormalTok{AppliedRate}
\end{Highlighting}
\end{Shaded}

\subsection{Merge data}\label{merge-data}

\begin{Shaded}
\begin{Highlighting}[]
\CommentTok{\#Combining the individual aggregated data set into one central table with a merged \textquotesingle{}Cell\textquotesingle{} variable}
\NormalTok{combine }\OtherTok{\textless{}{-}} \FunctionTok{merge}\NormalTok{(soyharvest17agg, cornharvest18agg, }\AttributeTok{by =} \StringTok{"cell"}\NormalTok{)}
\NormalTok{combine }\OtherTok{\textless{}{-}} \FunctionTok{merge}\NormalTok{(combine, cornseed18agg, }\AttributeTok{by =} \StringTok{"cell"}\NormalTok{)}
\NormalTok{combine }\OtherTok{\textless{}{-}} \FunctionTok{merge}\NormalTok{(combine, soyharvest19agg, }\AttributeTok{by =} \StringTok{"cell"}\NormalTok{)}
\NormalTok{combine }\OtherTok{\textless{}{-}} \FunctionTok{merge}\NormalTok{(combine, cornharvest20agg, }\AttributeTok{by =} \StringTok{"cell"}\NormalTok{)}
\end{Highlighting}
\end{Shaded}

\begin{verbatim}
## Warning in merge.data.frame(combine, cornharvest20agg, by = "cell"): column
## names 'Yield.x', 'Yield.y' are duplicated in the result
\end{verbatim}

\begin{Shaded}
\begin{Highlighting}[]
\NormalTok{Combined }\OtherTok{\textless{}{-}} \FunctionTok{merge}\NormalTok{(combine, cornseed20agg, }\AttributeTok{by =} \StringTok{"cell"}\NormalTok{)}
\end{Highlighting}
\end{Shaded}

\begin{verbatim}
## Warning in merge.data.frame(combine, cornseed20agg, by = "cell"): column names
## 'Yield.x', 'Yield.y' are duplicated in the result
\end{verbatim}

To help with interpreting and aligning the data to make it easier for
comparison we will normalize the data using the percent-of-mean. For
this approach we will observe i in year j, the normalized value is
defined as y\_ij\^{}*=100×y\_ij/y ‾\emph{(⋅j) , where y ‾}(⋅j ) swil be
the mean of bu/acre that was given or the arithmetic mean of all
observations for that variable in year j. This method will convert the
values to percentages relative to the yearly average, allowing
meaningful comparisons across crops with different units. This
normalization within year will preserve spatial variation while removing
scale differences across the dataset. By multiplying the values by 100
we can indicate whether there is above-average or below-average yield
performance. While not the only way for normalization, it is best for
making the results easier to interpret spatially and more suitable for
causal analysis involving agricultural decisions where exact values are
less important than if performance is exceeding or fewer than the
projection.

\begin{Shaded}
\begin{Highlighting}[]
\CommentTok{\#creating a function to make it easier to bring the percent to every measurement that we want to look at.we want to }
\NormalTok{percentseeding }\OtherTok{\textless{}{-}} \ControlFlowTok{function}\NormalTok{(x) \{}
    \DecValTok{100} \SpecialCharTok{*}\NormalTok{ x }\SpecialCharTok{/} \FunctionTok{mean}\NormalTok{(x, }\AttributeTok{na.rm =} \ConstantTok{TRUE}\NormalTok{)}
\NormalTok{\}}
\NormalTok{percentsoybean }\OtherTok{\textless{}{-}} \ControlFlowTok{function}\NormalTok{(x) \{}
    \DecValTok{100} \SpecialCharTok{*}\NormalTok{ x }\SpecialCharTok{/} \DecValTok{60}
\NormalTok{\}}
\NormalTok{percentcorn }\OtherTok{\textless{}{-}} \ControlFlowTok{function}\NormalTok{(x) \{}
    \DecValTok{100} \SpecialCharTok{*}\NormalTok{ x }\SpecialCharTok{/} \DecValTok{180}
\NormalTok{\}}
\CommentTok{\#Bringing the percent function into every variable that we intend to look at }
\NormalTok{soyharvest17}\SpecialCharTok{$}\NormalTok{Yieldnorm }\OtherTok{\textless{}{-}} \FunctionTok{percentsoybean}\NormalTok{(soyharvest17}\SpecialCharTok{$}\NormalTok{Yield)}
\NormalTok{cornharvest18}\SpecialCharTok{$}\NormalTok{Yieldnorm }\OtherTok{\textless{}{-}} \FunctionTok{percentcorn}\NormalTok{(cornharvest18}\SpecialCharTok{$}\NormalTok{Yield)}
\NormalTok{cornseed18}\SpecialCharTok{$}\NormalTok{AppliedRatenorm }\OtherTok{\textless{}{-}} \FunctionTok{percentseeding}\NormalTok{(cornseed18}\SpecialCharTok{$}\NormalTok{AppliedRate)}
\NormalTok{soyharvest19}\SpecialCharTok{$}\NormalTok{Yieldnorm }\OtherTok{\textless{}{-}} \FunctionTok{percentsoybean}\NormalTok{(soyharvest19}\SpecialCharTok{$}\NormalTok{Yield)}
\NormalTok{cornharvest20}\SpecialCharTok{$}\NormalTok{Yieldnorm }\OtherTok{\textless{}{-}} \FunctionTok{percentcorn}\NormalTok{(cornharvest20}\SpecialCharTok{$}\NormalTok{Yield)}
\NormalTok{cornseed20}\SpecialCharTok{$}\NormalTok{AppliedRatenorm }\OtherTok{\textless{}{-}} \FunctionTok{percentseeding}\NormalTok{(cornseed20}\SpecialCharTok{$}\NormalTok{AppliedRate)}
\end{Highlighting}
\end{Shaded}

\begin{Shaded}
\begin{Highlighting}[]
\CommentTok{\# We will aggregate the new data to help with making the data easier to read when we build the pairs plot }
\NormalTok{soyharvest17normagg }\OtherTok{\textless{}{-}} \FunctionTok{aggregate}\NormalTok{(Yieldnorm }\SpecialCharTok{\textasciitilde{}}\NormalTok{ cell, }\AttributeTok{data =}\NormalTok{ soyharvest17, mean)}
\NormalTok{soyharvest17normagg}\SpecialCharTok{$}\NormalTok{Y17 }\OtherTok{\textless{}{-}}\NormalTok{ soyharvest17normagg}\SpecialCharTok{$}\NormalTok{Yieldnorm}
\NormalTok{cornharvest18normagg }\OtherTok{\textless{}{-}} \FunctionTok{aggregate}\NormalTok{(Yieldnorm }\SpecialCharTok{\textasciitilde{}}\NormalTok{ cell, }\AttributeTok{data =}\NormalTok{ cornharvest18, mean)}
\NormalTok{cornharvest18normagg}\SpecialCharTok{$}\NormalTok{Y18 }\OtherTok{\textless{}{-}}\NormalTok{ cornharvest18normagg}\SpecialCharTok{$}\NormalTok{Yieldnorm}
\NormalTok{cornseed18normagg }\OtherTok{\textless{}{-}} \FunctionTok{aggregate}\NormalTok{(AppliedRatenorm }\SpecialCharTok{\textasciitilde{}}\NormalTok{ cell, }\AttributeTok{data =}\NormalTok{ cornseed18, mean)}
\NormalTok{cornseed18normagg}\SpecialCharTok{$}\NormalTok{AR18 }\OtherTok{\textless{}{-}}\NormalTok{ cornseed18normagg}\SpecialCharTok{$}\NormalTok{AppliedRatenorm}
\NormalTok{soyharvest19normagg }\OtherTok{\textless{}{-}} \FunctionTok{aggregate}\NormalTok{(Yieldnorm }\SpecialCharTok{\textasciitilde{}}\NormalTok{ cell, }\AttributeTok{data =}\NormalTok{ soyharvest19, mean)}
\NormalTok{soyharvest19normagg}\SpecialCharTok{$}\NormalTok{Y19 }\OtherTok{\textless{}{-}}\NormalTok{ soyharvest19normagg}\SpecialCharTok{$}\NormalTok{Yieldnorm}
\NormalTok{cornharvest20normagg }\OtherTok{\textless{}{-}} \FunctionTok{aggregate}\NormalTok{(Yieldnorm }\SpecialCharTok{\textasciitilde{}}\NormalTok{ cell, }\AttributeTok{data =}\NormalTok{ cornharvest20, mean)}
\NormalTok{cornharvest20normagg}\SpecialCharTok{$}\NormalTok{Y20 }\OtherTok{\textless{}{-}}\NormalTok{ cornharvest20normagg}\SpecialCharTok{$}\NormalTok{Yieldnorm}
\NormalTok{cornseed20normagg }\OtherTok{\textless{}{-}} \FunctionTok{aggregate}\NormalTok{(AppliedRatenorm }\SpecialCharTok{\textasciitilde{}}\NormalTok{ cell, }\AttributeTok{data =}\NormalTok{ cornseed20 , mean)}
\NormalTok{cornseed20normagg}\SpecialCharTok{$}\NormalTok{AR20 }\OtherTok{\textless{}{-}}\NormalTok{ cornseed20normagg}\SpecialCharTok{$}\NormalTok{AppliedRatenorm}
\end{Highlighting}
\end{Shaded}

Combining the data sets together to help with comparison between the
yields and seeding.

\begin{Shaded}
\begin{Highlighting}[]
\NormalTok{combinenorm }\OtherTok{\textless{}{-}} \FunctionTok{merge}\NormalTok{(soyharvest17normagg, cornharvest18normagg, }\AttributeTok{by =} \StringTok{"cell"}\NormalTok{)}
\NormalTok{combinenorm }\OtherTok{\textless{}{-}} \FunctionTok{merge}\NormalTok{(combinenorm, cornseed18normagg, }\AttributeTok{by =} \StringTok{"cell"}\NormalTok{)}
\NormalTok{combinenorm }\OtherTok{\textless{}{-}} \FunctionTok{merge}\NormalTok{(combinenorm, soyharvest19normagg, }\AttributeTok{by =} \StringTok{"cell"}\NormalTok{)}
\NormalTok{combinenorm }\OtherTok{\textless{}{-}} \FunctionTok{merge}\NormalTok{(combinenorm, cornharvest20normagg, }\AttributeTok{by =} \StringTok{"cell"}\NormalTok{)}
\end{Highlighting}
\end{Shaded}

\begin{verbatim}
## Warning in merge.data.frame(combinenorm, cornharvest20normagg, by = "cell"):
## column names 'Yieldnorm.x', 'Yieldnorm.y' are duplicated in the result
\end{verbatim}

\begin{Shaded}
\begin{Highlighting}[]
\NormalTok{CombinedNorm }\OtherTok{\textless{}{-}} \FunctionTok{merge}\NormalTok{(combinenorm, cornseed20normagg, }\AttributeTok{by =} \StringTok{"cell"}\NormalTok{)}
\end{Highlighting}
\end{Shaded}

\begin{verbatim}
## Warning in merge.data.frame(combinenorm, cornseed20normagg, by = "cell"):
## column names 'Yieldnorm.x', 'Yieldnorm.y' are duplicated in the result
\end{verbatim}

\begin{itemize}
\tightlist
\item
  Here we are displaying how the variables that we created in the grid
  will help define how we produce the results and findings.It's a great
  visual to show connection and relationship of the variables.
\end{itemize}

\begin{Shaded}
\begin{Highlighting}[]
\ControlFlowTok{if}\NormalTok{ (}\SpecialCharTok{!}\FunctionTok{requireNamespace}\NormalTok{(}\StringTok{"BiocManager"}\NormalTok{, }\AttributeTok{quietly =} \ConstantTok{TRUE}\NormalTok{))}
    \FunctionTok{install.packages}\NormalTok{(}\StringTok{"BiocManager"}\NormalTok{)}
\FunctionTok{install.packages}\NormalTok{(}\StringTok{"bnlearn"}\NormalTok{)}

\NormalTok{BiocManager}\SpecialCharTok{::}\FunctionTok{install}\NormalTok{(}\StringTok{"Rgraphviz"}\NormalTok{)}
\end{Highlighting}
\end{Shaded}

\begin{Shaded}
\begin{Highlighting}[]
\FunctionTok{library}\NormalTok{(bnlearn)}
\end{Highlighting}
\end{Shaded}

\begin{verbatim}
## Warning: package 'bnlearn' was built under R version 4.5.2
\end{verbatim}

\begin{Shaded}
\begin{Highlighting}[]
\NormalTok{modela.dag }\OtherTok{\textless{}{-}} \FunctionTok{model2network}\NormalTok{(}\StringTok{"[Y17][AR18|Y17][Y18|AR18:Y17]"}\NormalTok{)}
\NormalTok{fita }\OtherTok{=} \FunctionTok{bn.fit}\NormalTok{(modela.dag, Combined[,}\FunctionTok{c}\NormalTok{(}\StringTok{\textquotesingle{}Y17\textquotesingle{}}\NormalTok{,}\StringTok{\textquotesingle{}AR18\textquotesingle{}}\NormalTok{,}\StringTok{\textquotesingle{}Y18\textquotesingle{}}\NormalTok{)])}

\NormalTok{strengtha }\OtherTok{\textless{}{-}} \FunctionTok{arc.strength}\NormalTok{(modela.dag, Combined[,}\FunctionTok{c}\NormalTok{(}\StringTok{\textquotesingle{}Y17\textquotesingle{}}\NormalTok{,}\StringTok{\textquotesingle{}AR18\textquotesingle{}}\NormalTok{,}\StringTok{\textquotesingle{}Y18\textquotesingle{}}\NormalTok{)])}
\FunctionTok{strength.plot}\NormalTok{(modela.dag, strengtha)}
\end{Highlighting}
\end{Shaded}

\begin{verbatim}
## Loading required namespace: Rgraphviz
\end{verbatim}

\pandocbounded{\includegraphics[keepaspectratio]{Ben.Loos.Final_files/figure-latex/unnamed-chunk-9-1.pdf}}

\begin{Shaded}
\begin{Highlighting}[]
\NormalTok{modelb.dag }\OtherTok{\textless{}{-}} \FunctionTok{model2network}\NormalTok{(}\StringTok{"[Y19][AR20|Y19][Y20|AR20:Y19]"}\NormalTok{)}
\NormalTok{fitv }\OtherTok{=} \FunctionTok{bn.fit}\NormalTok{(modelb.dag, Combined[,}\FunctionTok{c}\NormalTok{(}\StringTok{\textquotesingle{}Y19\textquotesingle{}}\NormalTok{,}\StringTok{\textquotesingle{}AR20\textquotesingle{}}\NormalTok{,}\StringTok{\textquotesingle{}Y20\textquotesingle{}}\NormalTok{)])}

\NormalTok{strengthb }\OtherTok{\textless{}{-}} \FunctionTok{arc.strength}\NormalTok{(modelb.dag, Combined[,}\FunctionTok{c}\NormalTok{(}\StringTok{\textquotesingle{}Y19\textquotesingle{}}\NormalTok{,}\StringTok{\textquotesingle{}AR20\textquotesingle{}}\NormalTok{,}\StringTok{\textquotesingle{}Y20\textquotesingle{}}\NormalTok{)])}
\FunctionTok{strength.plot}\NormalTok{(modelb.dag, strengthb)}
\end{Highlighting}
\end{Shaded}

\pandocbounded{\includegraphics[keepaspectratio]{Ben.Loos.Final_files/figure-latex/unnamed-chunk-9-2.pdf}}

\begin{Shaded}
\begin{Highlighting}[]
\NormalTok{modelc.dag }\OtherTok{\textless{}{-}} \FunctionTok{model2network}\NormalTok{(}\StringTok{"[Y17][AR18|Y17][Y18|AR18:Y17][Y19|Y17:AR18:Y18][AR20|Y19][Y20|AR20:Y19]"}\NormalTok{)}
\NormalTok{fitc }\OtherTok{=} \FunctionTok{bn.fit}\NormalTok{(modelc.dag, Combined[,}\FunctionTok{c}\NormalTok{(}\StringTok{\textquotesingle{}Y17\textquotesingle{}}\NormalTok{,}\StringTok{\textquotesingle{}AR18\textquotesingle{}}\NormalTok{,}\StringTok{\textquotesingle{}Y18\textquotesingle{}}\NormalTok{,}\StringTok{\textquotesingle{}Y19\textquotesingle{}}\NormalTok{,}\StringTok{\textquotesingle{}AR20\textquotesingle{}}\NormalTok{,}\StringTok{\textquotesingle{}Y20\textquotesingle{}}\NormalTok{)])}

\NormalTok{strengthc }\OtherTok{\textless{}{-}} \FunctionTok{arc.strength}\NormalTok{(modelc.dag, Combined[,}\FunctionTok{c}\NormalTok{(}\StringTok{\textquotesingle{}Y17\textquotesingle{}}\NormalTok{,}\StringTok{\textquotesingle{}AR18\textquotesingle{}}\NormalTok{,}\StringTok{\textquotesingle{}Y18\textquotesingle{}}\NormalTok{,}\StringTok{\textquotesingle{}Y19\textquotesingle{}}\NormalTok{,}\StringTok{\textquotesingle{}AR20\textquotesingle{}}\NormalTok{,}\StringTok{\textquotesingle{}Y20\textquotesingle{}}\NormalTok{)])}
\FunctionTok{strength.plot}\NormalTok{(modelc.dag, strengthc)}
\end{Highlighting}
\end{Shaded}

\pandocbounded{\includegraphics[keepaspectratio]{Ben.Loos.Final_files/figure-latex/unnamed-chunk-9-3.pdf}}
\#\#Pair Plots \#Original using pair plot to showcase how the data
columns represent the yield variables

\begin{Shaded}
\begin{Highlighting}[]
\FunctionTok{pairs}\NormalTok{(Combined[, }\FunctionTok{c}\NormalTok{(}\StringTok{\textquotesingle{}Y17\textquotesingle{}}\NormalTok{, }\StringTok{\textquotesingle{}AR18\textquotesingle{}}\NormalTok{, }\StringTok{\textquotesingle{}Y18\textquotesingle{}}\NormalTok{, }\StringTok{\textquotesingle{}Y19\textquotesingle{}}\NormalTok{, }\StringTok{\textquotesingle{}AR20\textquotesingle{}}\NormalTok{, }\StringTok{\textquotesingle{}Y20\textquotesingle{}}\NormalTok{)], }\AttributeTok{main =} \StringTok{"Pairs Plot of Aggregated Yield Variables"}\NormalTok{)}
\end{Highlighting}
\end{Shaded}

\pandocbounded{\includegraphics[keepaspectratio]{Ben.Loos.Final_files/figure-latex/unnamed-chunk-10-1.pdf}}
\#Normalized Plot

\begin{Shaded}
\begin{Highlighting}[]
\FunctionTok{pairs}\NormalTok{(CombinedNorm[, }\FunctionTok{c}\NormalTok{(}\StringTok{\textquotesingle{}Y17\textquotesingle{}}\NormalTok{, }\StringTok{\textquotesingle{}AR18\textquotesingle{}}\NormalTok{, }\StringTok{\textquotesingle{}Y18\textquotesingle{}}\NormalTok{, }\StringTok{\textquotesingle{}Y19\textquotesingle{}}\NormalTok{,}\StringTok{\textquotesingle{}AR20\textquotesingle{}}\NormalTok{,}\StringTok{\textquotesingle{}Y20\textquotesingle{}}\NormalTok{)],}
    \AttributeTok{main =} \StringTok{"Pairs Plot of Normalized Yield and Seeding Data"}\NormalTok{)}
\end{Highlighting}
\end{Shaded}

\pandocbounded{\includegraphics[keepaspectratio]{Ben.Loos.Final_files/figure-latex/unnamed-chunk-11-1.pdf}}

The output of the analysis shows what the use of normalization looks for
the seeding plots. As we can see in the results there is not a huge
difference what we see from the original data and the new normalized
data set. This could indicate that there are not huge fluctuation within
the variables that can change the data .By aggregating yield and seeding
data into consistent spatial grid cells and normalizing values across
years, the analysis separates management effects from inherent field
productivity. Including historical yield allows us to distinguish
whether low corn yields are primarily associated with reduced seeding
rates or reflect persistently low-yielding areas of the field. This
structure supports causal inference by accounting for prior productivity
when evaluating the effect of seeding rate on yield.

\end{document}
